\documentclass[10pt,a4paper,oneside]{article}
\usepackage[latin1]{inputenc}
\usepackage[english]{babel}
\usepackage{amsmath}
\usepackage{amsfonts}
\usepackage{amssymb}
\usepackage[T1]{fontenc} %use different encoding (copy from pdf is now possible}
\usepackage{fullpage} %small margins
\usepackage{color}
\definecolor{light-gray}{gray}{0.85}
\usepackage{listings} %sourcecode
\lstset{
    numbers=left,
    breaklines=true,
    backgroundcolor=\color{light-gray},
    tabsize=2,
    basicstyle=\ttfamily,
}
\begin{document}\title{3.18}\author{Andrew Lee}\maketitle{}
\textrm{You are given a binary tree }T = (V, E ) \textrm{(in adjacency list format), along with a designated root node }r \in V\textrm{. Recall that }u \textrm{ is said to be an ancestor of }v\textrm{ in the rooted tree, if the path from }r \textrm{ to } v \in \textrm{ T passes through }u\textrm{.}\\\\
\textrm{You wish to preprocess the tree so that queries of the form ``is u an ancestor of v?'' can be answered}\\\textrm{in constant time. The preprocessing itself should take linear time. How can this be done?}\\\\\\
\textrm{Doing a depth first search on the tree, starting with some root, we record the pre and post numbers}\\\textrm{of every vertex on the graph. Then, we can say definitively that u is the ancestor of v if }\\pre(u) < pre(v) < post(v) < post(u)\textrm{.}

\end{document}