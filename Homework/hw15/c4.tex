\documentclass[10pt,a4paper,oneside]{article}
\usepackage[latin1]{inputenc}
\usepackage[english]{babel}
\usepackage{amsmath}
\usepackage{amsfonts}
\usepackage{amssymb}
\usepackage[T1]{fontenc} %use different encoding (copy from pdf is now possible}
\usepackage{fullpage} %small margins
\usepackage{color}
\definecolor{light-gray}{gray}{0.85}
\usepackage{listings} %sourcecode
\lstset{
    numbers=left,
    breaklines=true,
    backgroundcolor=\color{light-gray},
    tabsize=2,
    basicstyle=\ttfamily,
}
\begin{document}\title{3.16}\author{Andrew Lee}\maketitle{}

\textrm{Suppose a CS curriculum consists of \textit{n} courses, all of them mandatory. The}\\\textrm{prerequisite graph \textit{G} has a node for each course, and an edge from course \textit{v} to}\\\textrm{course \textit{w} if and only if \textit{v} is a prerequisite for \textit{w}. Find an algorithm that works}\\\textrm{directly with this graph representation, and computes the minimum number of}\\\textrm{semesters necessary to complete the curriculum (assume that a student can take}\\\textrm{any number of courses in one semester). The running time of your algorithm}\\\textrm{should be linear.}\\\\
\textrm{The first thing to note is that this will be a directed graph since we need to specify }\\\textrm{a pre-requisite, class u for class v, it is not true that}\\\textrm{class v will be a prerequisite for class u. Furthermore, since ``a student can take any number of courses in one semester'', we can start at all the classes that do not have a requirement or rather vertices}\\\textrm{that are only sources. vertex, and simultaneously traverse all the adjacency lists of vertices at every stage.}\\\\\textrm{The specific search algorithm for this problem will be breadth first search,}\\\textrm{which specializes in being able to find all the possible vertices within n steps (edges)}\\\textrm{from the root (or roots). This behavior is desirable because of the condition that we can traverse}\\\textrm\{multiple edges at once, and the speed at which BFS works becomes better more edges can be traversed.}

\lstinputlisting[tabsize=2]{curriculum.py}

\\\textrm{The reason why this program is linear is because we will wait at most N steps for the computation}\\\textrm{ to complete, but because we are allowed to take ``multiple classes'' this will most likely be}\\\textrm{significantly faster than just taking one step at a time. In fact, although }T(n)=\bigotimes(n),\ T(n)=\Omega(1)\\\textrm{because in the best case scenario, with no prerequisites for classes, we can add all the classes in one ``step''.}
\end{document}